\documentclass[12pt]{article}
\usepackage{geometry}
\geometry{letterpaper}
\usepackage{graphicx}
\usepackage{setspace}
\usepackage{amssymb}
\usepackage{amsmath}
\usepackage{epstopdf}
\usepackage[numbers]{natbib}
\usepackage[unicode=true]{hyperref}
\hypersetup{breaklinks=true,
            pdfauthor={},
            pdftitle={},
            colorlinks=true,
            citecolor=blue,
            urlcolor=blue,
            linkcolor=magenta,
            pdfborder={0 0 0}}

\setlength{\parindent}{0ex}
\setlength{\parskip}{2em}

\title{Scaling the species abundance distribution: annotated
  bibliography}

\author{A. J. Rominger}

\begin{document}
\maketitle

\citet{bordaDeAgua2012} focus on upscaling the SAD by estimating its
momenst at a smaller scale and then scaling those moments up.  They
re-iterate, as do many papers, that individuals are aggregated and so
scaling the SAD does not preserve its shape---although it would under
Poisson spatial process.

\citet{chisholm2009} relate the shape and size of a plot to the $m$
immigration parameter of NTB. In the process they derive scaling for
the SAD which they find goes from ZSM at small scale to logseries at
large scale. They do not discuss the same nuance as
\citet{rosindell2013}

\citet{green2007} present a sampling theory for SADs based on random
sampling (Poisson) or aggregated sampling (negative binomial). They
confirm that random sampling preserves the shape of the SAD while
aggregated sampling can interestingly often lead to something close to
Fisher-looking.

\citet{harte1999} derive self-similar scaling of SAR and SAD
\citep[but see][who says they did it wrong for SAD]{pueyo2006}.

\citet{hubbell2008} show power law scaling between nearest neighbor
distance and rank of distance (this all in an effort to estimate
number and abundance of tree species in Amazonia). Power exponant
$>0.5$ indicates non-Poisson aggregation.  Some of these results might
also show up in Harte's MaxEnt book.

\citet{mcgill2003} argues that log right-skewed distributins,
e.g. Fisher logseries, are sampling artifacts, that taking a small
subsample of any large community will result in log right-skew. Does
does this mostly with simulation without considering spatial
aggregation; but he also uses BCI, without much discussing it. Also
misses the fact that Poisson sampling (as he does in ``Model I'') will
preserve the parametric form of the SAD, so skewness might not be the
thing to look at, rather the parametric form could be more
informative.

\citet{myers2015} use burnt and unburnt plots to look at how
disturbance influences turnover. Use spatial and non-spatial
rarefaction to conclude that disturbance changes $\beta$-diversity
patterns but not underlying processes.

\citet{odwyer2017} provide an analytical solution to spatially
explicit neutral theory and the spatial scaling of the SAD. They
confirm work by \citet{rosindell2013}, including the scale collapse of
the SAD when speciation rate is small.  Could such scale collapse
explain near universal success of Fisher at small scale?  Also while
they confirm the result of \citet{rosindell2013} that at largest scale
you get a log series, at intermediate, but still quite large, you get
something hump-shaped on a log scale.  This is also consistent with
\citet{rosindell2013} result that singletons first increase, then
decrease, then increase again.  Singletons increasing at largest scale
is probably due to point mutation speciation---would be interesting to
investigate with protracted speciation model.

\citet{plotkin2000} show that power-law SAR is wrong from scales of 1
m$^2$--50 ha (in the process of estimating species richness for
tropical trees).  Also show that upscaling SADs (various models) to
get species estimates generally lead to over-estimate of S at 50 ha
scale for Pasoh.  Must be assuming poisson spatial process, because no
other process is mentioned.

\citet{pueyo2006} re-derives self-similar SAR and SAD, showing a
power-law SAD leads to a power-law SAR.  He does not consider how SAR
will scale with area, we're left to assume it's the same shape
everywhere.

\citet{rosindell2013} explore scaling of the NTB.  They find a
tri-phasic accumulation of singleton species---first increasing with
area, then decreasing, then increasing again.  This pattern depends on
on the biogeoraphic range of the species considered---a qunatity
determined by specieation rate and dispersal kernal.  This finding
ties in with scale collapes of tri-phasic SAR (Harte and Storch).

\citet{stegen2013} show that $\beta$-diversity is driven both by
sampling and deterministic interactions with environment---namely
heterogeneity and primary productivity.  The do this with BBS.


\bibliographystyle{plainnat}
\bibliography{sadScaling.bib}

\end{document}



